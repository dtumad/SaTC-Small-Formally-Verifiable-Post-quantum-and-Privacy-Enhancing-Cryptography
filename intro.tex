\section{Introduction}
\label{sec:intro}

In the last decade, many formerly-theoretical cryptographic constructions have been deployed in widely-used and high-value applications, ranging from messaging, epidemic things, and device tracking, to voting and passport verification, to financial and blockchain technology.  More of these systems are expected to be deployed in the coming years as systems using so-called postquantum and homomorphic encryption are deployed.  

Because of the obviously high consequences of failure and the difficulty of protecting encrypted information after it has been transmitted, cryptographers since the 1980s have recognized the desirability of {\em provable security}, where a proof is given that certain classes of attacks cannot succeed against a cryptographic algorithm, under basic mathematical assumptions; and most algorithms in the cryptographic literature are accompanied by such proofs.  However, proofs are subject to failure, OAEP, Plonk, Elgamal signatures, etc. This has led to the desire for more automation in the verification of proofs.

In the past 20 years, several systems for formally verifying the security of cryptographic algorithms have been proposed, symbolic vs axiomatic vs foundational \textbf{FIXME}.  Only foundational systems directly address the problem of verifying the security proofs of the underlying algorithms.  However, the foundational systems to date fall well short of verifying the security of the systems deployed today; for instance, the state-of-the-art cryptHOL system cannot model ``programmable oracles'' used in the security proofs of OAEP, the Fujisaki-Okamoto transform, or even the long-used Schnorr signature scheme.

To bridge the gap between the capability of existing proof systems, this project will build on our recent work developing a new foundational framework for formally verifiable cryptographic proofs, \vcvio.  \vcvio models cryptographic computations and security proofs as a shallow embedding in the Lean proof assistant and can overcome the shortcomings of previous frameworks.  The project will address the urgent need for formally verifiable security proofs of recent and soon-to-be-deployed cryptographic algorithms through three primary objectives.  First, we will extend \vcvio with modules and tactics that will make developing proofs easier for other cryptographer.  Second, we will develop formally verifiable proofs for the recently announced postquantum cryptographic primitives FIXME.  Finally, as an additional application area, we will apply \vcvio to investigate and prove the security of cryptographic algorithms in the area of end-to-end encrypted messaging systems.

FIXME Intellectual merit and broader impact important stuff.