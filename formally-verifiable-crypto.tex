\documentclass[12pt]{article} 
% The Proposal and Award Policies and Procedures Guide
% (PAPPG: https://www.nsf.gov/publications/pub_summ.jsp?ods_key=pappg)
% mandates, in Chapter 2, section B.2, that the main text should have a font size no less
% than 11 points for *most* typefaces (including Computer Modern Roman and Times (new) Roman).
% 
% Actually, Helvetica (a.k.a. Arial), Palatino and Courier New can drop
% to 10 point font size, according to the PAPPG, but be aware:
% 10-point fonts (whatever the typeface) will promote reader fatigue.
% Reader fatigue never works to the author's advantage.
%
%
% This sample  assumes the pdflatex chain (not the traditional latex | dvips | ps2pdf chain).
% It should function in the traditional chain if you comment out the hyperref package lines.
% 
%\usepackage{times} % uncomment for Times Roman;
%                   % replace "times" with other font names, as desired. 
\usepackage{bm} % If you want bold math, but without the bloat of AMS packages
\usepackage{amsfonts}
\usepackage{amsmath}
\usepackage{color} % doh
\usepackage{hyperref} % LaTeX cross references become hyperlinks in pdf output
\usepackage{graphicx} % For figure insertion
\usepackage{vmargin} % To set page size
\usepackage{xspace}
%
% Hyperref stuff: WARNING: don't use \href{link}{clickable text} in the Project Description,
% as Research.gov doesn't like it. Likewise, avoid any text containing 
% http:, https:, mailto:, etc.
%
\hypersetup{colorlinks=true,linkcolor=blue,urlcolor=blue}
%
% vmargin parameters:
% The Proposal and Award Policies and Procedures Guide
% (PAPPG: https://www.nsf.gov/publications/pub_summ.jsp?ods_key=pappg)
% mandates, in Chapter 2, section B.2, that margins must be at least 1 inch in all directions.
%
\setpapersize{USletter}
% \setmarginsrb{〈leftmargin〉}{〈topmargin〉}{〈rightmargin〉}{〈bottommargin〉}%{〈headheight〉}{〈headsep〉}{〈footheight〉}{〈footskip〉}
\setmarginsrb{1in}{1in}{1in}{1in}{0pt}{0mm}{0pt}{0mm}

%
% Pseudodot: break up "clickable" strings like "Data.gov" which you can
% enter as Data\pseudodot{}gov in your file.
% Use this trick in case research.gov misreads text such as "Data.gov" as a clickable hyperlink
% (even when it isn't).
%
\usepackage{fontspec}
\setmonofont{FreeMono}


% Syntax highlighting and formatting (using lean4.py file)
\usepackage{minted}
\newmintinline[lean]{'lean4.py:Lean4Lexer -x'}{fontsize=\small}
\newmintinline[leanfootnote]{'lean4.py:Lean4Lexer -x'}{fontsize=\footnotesize}
\newminted[leancode]{'lean4.py:Lean4Lexer -x'}{fontsize=\small}
\usemintedstyle{vs}

% Minted can't put boxes if we undefine boxes from existing *Tapping Head*
\renewcommand{\fcolorbox}[4][]{#4}
% Added just to list forking lemma

\newcommand{\vcvio}{{\sf VCVio}\xspace}
\newcommand{\pseudodot}{{\lower 2.4pt\hbox{$\cdot$}}}
%\def\pseudodot{{\lower 2.4pt\hbox{$\cdot$}}} %% This is the PlainTeX version.
%
% On with the show...
%
\begin{document}
%
% Remove page numbers---Research.gov will paginate the document instead
%
\pagestyle{empty} 
%
% Line spacing: 
% The Proposal and Award Policies and Procedures Guide
% (PAPPG: https://www.nsf.gov/publications/pub_summ.jsp?ods_key=pappg)
% mandates, in Chapter 2, section B.2, that the main text should have no more than
% six lines to the vertical inch. With 72 points per inch, the minimum line skip 
% would be 12 points.
%
\setlength{\baselineskip}{12.6pt} % in text mode
\setlength{\normalbaselineskip}{12.6pt} % in math mode

%
% Actual text follows
%

\section{Introduction}
\label{sec:intro}

In the last decade, many formerly-theoretical cryptographic constructions have been deployed in widely-used and high-value applications, ranging from messaging, epidemic things, and device tracking, to voting and passport verification, to financial and blockchain technology.  More of these systems are expected to be deployed in the coming years as systems using so-called postquantum and homomorphic encryption are deployed.  

Because of the obviously high consequences of failure and the difficulty of protecting encrypted information after it has been transmitted, cryptographers since the 1980s have recognized the desirability of {\em provable security}, where a proof is given that certain classes of attacks cannot succeed against a cryptographic algorithm, under basic mathematical assumptions; and most algorithms in the cryptographic literature are accompanied by such proofs.  However, proofs are subject to failure, OAEP, Plonk, Elgamal signatures, etc. This has led to the desire for more automation in the verification of proofs.

In the past 20 years, several systems for formally verifying the security of cryptographic algorithms have been proposed, symbolic vs axiomatic vs foundational \textbf{FIXME}.  Only foundational systems directly address the problem of verifying the security proofs of the underlying algorithms.  However, the foundational systems to date fall well short of verifying the security of the systems deployed today; for instance, the state-of-the-art cryptHOL system cannot model ``programmable oracles'' used in the security proofs of OAEP, the Fujisaki-Okamoto transform, or even the long-used Schnorr signature scheme.

To bridge the gap between the capability of existing proof systems, this project will build on our recent work developing a new foundational framework for formally verifiable cryptographic proofs, \vcvio.  \vcvio models cryptographic computations and security proofs as a shallow embedding in the Lean proof assistant and can overcome the shortcomings of previous frameworks.  The project will address the urgent need for formally verifiable security proofs of recent and soon-to-be-deployed cryptographic algorithms through three primary objectives.  First, we will extend \vcvio with modules and tactics that will make developing proofs easier for other cryptographer.  Second, we will develop formally verifiable proofs for the recently announced postquantum cryptographic primitives FIXME.  Finally, as an additional application area, we will apply \vcvio to investigate and prove the security of cryptographic algorithms in the area of end-to-end encrypted messaging systems.

FIXME Intellectual merit and broader impact important stuff.
\section{Background and Preliminary work}

\subsection{Previous frameworks and challenges}
A wide variety of frameworks have been developed to reason about cryptographic proofs in the computational setting, which all have different benefits and drawbacks depending on the specific use cases.
Some like EasyCrypt \cite{canetti2019easyuc}, CryptoVerif \cite{blanchet2007cryptoverif} or SSProve \cite{abate2021ssprove} take a protocol centered approach that focuses mainly on high-level functionality and behavior, often using a deeply-embedded language and proof system to represent and reason about computations.
Others like FCF \cite{FCF,petcher2015foundational}, CertiCrypt \cite{zanella2010formal}, or CryptHOL \cite{basin2020crypthol} take a more foundational approach, building all constructions from a small computing base with limited axiomatization.
Usually computations in this approach are represented by a shallow embedding into a more general proof assistant like Coq or Isabelle.
The foundational approach has significant benefits in terms of flexibility and extensibility, and allows for concrete security bounds,
however it usually provides less automation when writing proofs.

One important limitation with existing foundational systems is how they reason about the oracles available to a computation.
In FCF, for example, oracle access is specified by a single oracle with fixed input and output types. This makes it difficult to reason about situations where multiple (or varying) oracles are available, e.g. both a signing oracle and a random oracle.  Although dependent sum types for the domain and range of different oracles could be used, the specification of programs and reductions that use these oracles quickly becomes cumbersome, and cannot easily handle composition of programs that use different sets of oracles.
In CryptHOL, oracles are implemented as ``generative probabilistic values'' and require restructuring adversaries and experiments into a series of smaller computations that run between oracle calls.  Because adversaries cannot access the state of oracles, CryptHOL cannot easily express reductions in which oracles are reactively ``programmed'' or rewound to intermediate states by one adversary -- suach as an adversary that solves a hard computational problem -- in response to the queries or results output by another adversary, such as an adversary that outputs a signature forgery. 

\subsection{The VCVio framework}
\label{sec:vcvio}
Similar to FCF and CryptHOL, \vcvio uses a shallow embedding of cryptographic algorithms, reductions, and security definitions as functional programs in the Lean proof assistant.  Lean  is a dependently typed programming language and interactive proof assistant that enables computer verification of both mathematical proofs and properties of programs, similar to Coq or Isabelle.  

At a basic level Lean is a functional programming language, and its syntax is similar to languages like Haskell or OCaml.
The core logic of Lean is based on a type theory called the calculus of constructions (CIC), where every expression is a term and every term has a type, denoted \lean{(a : A)}.
The type of a function from type \lean{A} to type \lean{B} is written \lean{A → B}, and function terms are expressed using lambda notation:

\begin{leancode}
  def foo : ℕ → ℕ → ℕ := λ x y ↦ x * y + 1
\end{leancode}

Functions can be dependently typed, so the type of an argument (or the output type) can depend on previous arguments to the function.
For example the function \lean{List.cons} is polymorphic in the type of the list's elements, which we express by naming the type of the first argument:

\begin{leancode}
List.cons : {A : Type} → List A → A → List A
\end{leancode}

The curly braces indicate that when this function is called Lean should try to automatically determine the value of \lean{A} based on type inference. 
This allows us to write \lean{List.cons xs x} whether the elements of \lean{xs} are strings, natural numbers, or something else.

The standard algebraic data types are built into the main language.
Product types are written \lean{A × B} with canonical elements \lean{(a, b)} for \lean{a : A} and \lean{b : B}.
Sum types are written \lean{A ⊕ B} with canonical elements \lean{inl a} and \lean{inr b}.
The singleton type \lean{Unit} has a single canonical element \lean{()}, and the empty type \lean{⊥} has no elements.
We also have a type \lean{Bool}, with canonical elements \lean{true} and \lean{false}.

In order to be able to write and verify proofs, Lean also has a type \lean{Prop} to represent mathematical propositions, and we view the propositions in \lean{Prop} as themselves being types.
Under this framework, an element \lean{P : Prop} is a mathematical statement, and the "elements" of \lean{P} are the proofs of that statement.  While it's possible to write proofs as explicit terms in the language, lean also offers an interactive proof environment that allows proofs to be written via \textit{tactic programming}.
A tactic proof consists of a number of steps that modify the current "goal" to be proven, either solving it explicitly or generating new goal(s) that would suffice to complete the proof.
The initial goal is the statement being proved, and a proof is complete if no goals remain.
Many tactics are built in to the core language, but they can also be defined by users.
One particularly useful tactic is \lean{simp}, which takes in a list of equivalences, and attempts to replace any instances of the left hand side of the equality in the goal with the right hand side. 
As an example we can use it to show that if $x + y = z$ for some non-negative $y$, then either $x < z$ or $y$ is $0$:

\begin{leancode}
  example (x y z : ℝ) (hy : 0 ≤ y)
      (hz : x + y = z) : x < z ∨ y = 0 :=
    by simp [← hz, le_iff_lt_or_eq, hy]
\end{leancode}

\subsubsection{A Monad for Computations with Oracle Access} \label{OracleComp}
In order to reason about cryptographic protocols, we define a shallow embedding of computations with oracle access into the Lean type system, using a free monad to augment regular Lean functions with queries to oracles.

\paragraph{Specifying Oracles}
Before defining our model of computation we define a way to specify the set of oracles available to a computation, via a structure \lean{OracleSpec} consisting of an indexing set $\iota$ of available oracles and maps from indices to the domain and range types of the corresponding oracle:
\begin{leancode}
  structure OracleSpec where (ι : Type) (domain range : ι → Type)
\end{leancode}

Each element \lean{i : ι} of the indexing set corresponds to a unique oracle, and \lean{domain i} and \lean{range i} are the input and output types of the oracle corresponding to \lean{i}.
Note that we don't specify any behavior for these oracles; this should rather be seen as analogous to a type signature for them.
For instance we define \lean{singletonSpec T U} to represent access to
a single oracle with input type \lean{T} and output type \lean{U},
by using the singleton type \lean{Unit} for the indexing set, and have the \lean{domain} and \lean{range} functions return \lean{T} and \lean{U} respectively:
\begin{leancode}
  def singletonSpec (T U : Type) : OracleSpec :=
    (ι := Unit) (domain := λ () ↦ T) (range := λ () ↦ U)
\end{leancode}
We also define \lean{emptySpec : OracleSpec} to represent having no oracles at all.
We introduce the notation \lean{T →ₒ U} for the singleton spec and \lean{[]ₒ} for the empty spec.

In order to represent probabilistic computation we define two additional oracle sets.
Firstly \lean{coinSpec} that gives access to a single coin flipping oracle returning a \lean{Bool},
and secondly \lean{unifSpec} that gives access to an infinite family of oracles, indexed over \lean{ℕ}, where the $n$th oracle chooses a random value between $0$ and $n$ inclusively (represented by the type \lean{Fin (n + 1)} in Lean).
Since the input to these oracles is irrelevant, in both cases we use \lean{Unit} for the domain types:
\begin{leancode}
  def coinSpec : OracleSpec := Unit →ₒ Bool
  def unifSpec : OracleSpec :=
    (ι := ℕ) (domain := λ _ ↦ Unit) (range := λ n ↦ Fin (n + 1))
\end{leancode}

\paragraph{Representing Computations.}
We now define a way to represent computations via an inductive type \lean{OracleComp spec α}, where \lean{spec : OracleSpec} specifies what oracles the computation can make use of, and \lean{α} gives the type of the final output.
This type is a shallow embedding, so functions and expressions in this representation are just functions and expressions in Lean, augmented with the ability to call oracles.

The embedding is done as a monad, which is a common way to represent computations with side effects in languages where all computations are pure.
In our case the "side effects" being captured by the monad are queries to the oracles.  We define this as an inductive type with only two constructors.
The first is \lean{pure' α x} which represents returning an pure Lean value \lean{x : α}.
The other, \lean{queryBind' i t α oa}, represents querying oracle \lean{i} on input \lean{t} to get a result \lean{u} and then running the computation \lean{oa u}.  Explicitly:
\begin{leancode}
  inductive OracleComp (spec : OracleSpec) : (α : Type) → Type
    | pure' (α : Type) (x : α) : OracleComp spec α
    | queryBind' (i : spec.ι) (t : spec.domain i)
        (α : Type) (oa : spec.range i → OracleComp spec α) :
        OracleComp spec α
\end{leancode}
Note that the function \lean{oa} in the \lean{queryBind'} constructor is an arbitrary Lean function.
It can therefore perform arbitrary computations in the process of returning the continuation.
Most of the "interesting" behavior of a computation is captured in this function, with the monad structure simply adding the ability to represent query calls.

We define \lean{query i t} to be the computation that returns the result of querying oracle \lean{i} with \lean{t}, directly returning the result as a pure value after:
\begin{leancode}
  def query (i : spec.ι) (t : spec.domain i) : 
    OracleComp spec (spec.range i) :=
    queryBind' i t (spec.range i) (λ u ↦ pure' (spec.range i) x)
\end{leancode}
We define computations representing flipping a coin and selecting from a range as special cases of \lean{query}:
\begin{leancode}
  def coin : OracleComp coinSpec Bool := query () ()
  def $[0..n] : OracleComp unifSpec (Fin (n+1)) :=  query () n
\end{leancode}
We overload notation and write \lean{$ xs} for random selection from any type of collection.

The general monadic bind operation on this type, \lean{bind' α β oa ob} will represent running the computation \lean{oa} to get a result \lean{x : α}, then running \lean{ob} on input \lean{x} to get a result of type \lean{β}.
We define this by induction on the first computation.
If the first computation \lean{oa} is a pure value, we insert it directly into the remaining computation.
If the first computation is a query bound to a continuation, we move the second computation inside the continuation by a recursive call to \lean{bind'}.
\begin{leancode}
  def bind' (α β : Type) : OracleComp spec α →
      (α → OracleComp spec β) → OracleComp spec β
    | pure' α a, ob => ob a
    | queryBind' i t α oa, ob =>
        query_bind' i t β (λ u ↦ bind' α β (oa u) ob)
\end{leancode}
\lean{OracleComp spec} with the operations \lean{pure'} and \lean{bind'} then forms a monad, and the three monad laws can all be verified easily by induction.
We will generally use Lean's monad type class and its associated notation to write computations.
In particular we can write \lean|return a| for the pure operation, \lean|oa >>= ob| for the bind operation, and use \textit{do}-notation for sequencing larger computations, as in \lean{example : OracleComp coinSpec ℝ := do let b ← coin;} \lean{let b' ← coin;} \lean{ return (if b && b' then 3.14 else 1.61)}.

\paragraph{Sub-Specs and Type Coercions}
In order to combine sets of oracles we define an append operation on \lean{OracleSpec}, to represent having access to oracles in either of the two original specs. We make use of sum types for the indexing set, with the \lean{inl} and \lean{inr} functions used to index the first and second sets of oracles respectively.
The types of the domain and range at each index are defined by pattern matching on the provided index.
%:
%\begin{leancode}
%  def append (spec₁ spec₂ : OracleSpec) : OracleSpec where
%    ι := spec₁.ι ⊕ spec₂.ι
%    domain := λ i ↦ match i with
%      | inl i => spec₁.domain i
%      | inr i => spec₂.domain i
%    range := λ i ↦ match i with
%      | inl i => spec₁.range i
%      | inr i => spec₂.range i
%\end{leancode}
We introduce the notation \lean|spec ++ spec'| for this operation.

One major issue in this representation is that it gives no way to sequence or combine computations where one has only a subset of the oracles of another.
For example we can't bind \lean{coin} to an adversary that has access to both a coin oracle and random oracle, since our monad definition requires that the \lean{OracleSpec} remain fixed throughout a computation.
To solve this we create a system for automatically coercing the type of a computation to one with a larger set of oracles.
We do this using Lean's \lean{Coe A B} type-class, which specifies to Lean a way to automatically convert a value of type \lean{A} to one of type \lean{B} (e.g. \lean{Coe ℚ ℝ} allows a rational number to be viewed as an element of the reals).
Lean will automatically perform a type-class search for this whenever it fails to type check an expression, and apply a coercion if it finds one.  
The semantics we define in the next to sections are highly compatible with these coercions, and and our system is generally able to reduce a proof about a coerced computation down to an analogous proof about the original computation automatically.

\paragraph{Probability Semantics} \label{DistSemantics}
We next give a denotational semantics for \lean|OracleComp|, where the denotation is a probability distribution modeling the probability of getting specific outputs from the computation.  Specifically, we define a function \lean{evalDist} that maps computations with return type \lean{α} to probability mass functions on \lean{α}, represented by the type \lean{PMF α} (originally developed for use in the CryptoLib project \cite{CryptoLib}).
\lean{PMF} itself is a monad and the mapping is a morphism of monads (i.e. it respects \lean{pure} and \lean{bind}).
It's then possible to define a measure on \lean{α} induced by this \lean{PMF}, and to define the probability of some predicate \lean{p} holding after running a computation as the measure of the set \lean{{x | p x}}.

We will write \lean{[= x | oa]} for the probability associated to \lean{x} by \lean{evalDist oa}, which gives a simple characterization of the semantics in practice:
\begin{leancode}
[= x | return a] = if x = a then 1 else 0
[= y | oa >>= ob] = ∑ x, [= x | oa] * [= y | ob x]
[= u | query i t] = 1 / (spec.range i).card
\end{leancode}
We also define \lean{[p | oa]} to be the probability associated to the event \lean{p : α → Prop}.
If the event is instead a set \lean{e : Set α} then we can write \lean{[(· ∈ e) | oa]} for the probability that the output is a member of \lean{e}. 
We define \lean{support oa} to be the set of possible outputs of \lean{oa}, i.e. the set of \lean{x} such that \lean{[= x | oa] ≠ 0}.
We make heavy use of this when showing that an event has probability either \lean|1| or \lean|0|, allowing us to think about ``possible outputs'' rather than explicit probabilities.

\paragraph{Simulation Semantics} 
We now give an alternative semantics for \lean{OracleComp}, which provide a method for simulating the behavior of oracles.
The main construction is a function \lean{simulate} that recursively substitutes the queries to an oracle with a specified implementation (that may have access to a different set of oracles).
We allow the simulation to maintain some internal state, and augment the return value of the computation with the final state.
For example, simulating a random oracle will consist of maintaining an internal cache of input/output pairs as the state, and the replacement will substitute queries to include a check to the cache before responding.

In order to represent a procedure for simulating a computation, we define a type \lean{SimOracle} \lean{spec} \lean{spec'} \lean{σ} for an implementation of the oracles in \lean{spec} using a new set of oracles \lean{spec'}, where \lean{σ} is the type of an internal state shared throughout the simulation.
This is given by a function that takes an oracle input and returns a new computation that should be used to replace the query:
\begin{leancode}
  def SimOracle (sp : OracleSpec) (sp' : OracleSpec) (σ : Type) :=
    (i : sp.ι) → sp.domain i → σ → OracleComp sp' (sp.range i × σ)
\end{leancode}
We introduce the notation \lean{spec →[σ] spec'} for this type.

We then define a function \lean{simulate} that applies a simulation oracle to a computation, recursively replacing queries with the new computations, passing the updated state along before returning the final value of the computation and the final state value.
\begin{leancode}
  def simulate (so : spec →[σ] spec') :
      (oa : OracleComp spec α) → (s : σ) → OracleComp spec' (α × σ)
    | pure' α x, s => return (x, s)
    | query_bind' i t α oa, s => do let (u, s') ← so i t s;
        simulate so (oa u) s'
\end{leancode}

In order to represent more complex oracle implementations in a modular way, we define a number of ways to combine multiple simulation oracles.
To construct a combined simulation oracle for \lean{spec₁ ++ spec₂}
given individual simulation oracles for each, the simulation oracle pattern matches the oracle index it receives and forwards to the corresponding simulation oracle, maintaining independent states for each of them.
We define the \lean{compose} simulation to construct a simulation oracle that applies two simulation oracles in sequence, by simulating the simulation function of one with the other.  We will write \lean{++} and \lean{∘} for these two operations.

Besides just implementing oracle behavior, another important use case
is tracking some information about a computation.
For example we can easily implement simulation oracles \lean{loggingOracle} for logging the queries made by an oracle, \lean{cachingOracle} for logging fresh values, but returning old values if the input has been queried already, and \lean{randOracle} as the composition of a \lean{cachingOracle} and a \lean{unifOracle}.

\subsubsection{Cryptographic Protocols and Security Games} \label{sec:protocols}
In this section we give a simple abstraction layer that can be used to specify cryptographic primitives, protocols, and security games.
We focus here on using this to establish concrete security bounds, however it can be directly extended to the asymptotic setting.
We start by defining a type \lean{OracleAlg spec} that just contains a specification of how to simulate oracles in \lean{spec} using \lean{unifSpec}.
Here \lean{spec} should be thought of as some global oracles for a protocol (e.g. a random oracle), and the structure as containing the ``intended behavior'' of them.
We also assume an intended initial state (e.g. an empty cache):
\begin{leancode}
  structure OracleAlg (spec : OracleSpec) where
   (baseState : Type) (init_state : baseState)
   baseSimOracle : spec →[baseState] unifSpec
\end{leancode}
Given \lean{alg : OracleAlg spec}, we will write \lean{alg.exec oa} as shorthand for simulating \lean{oa} with the bundled oracle in \lean{alg}. 
This allows us to define the type of a cryptosystem in a way that is agnostic to the oracles that are available.

As an example we define the type of a signature protocol as extending this structure with keygen, sign, and verify functions:
\begin{leancode}
  structure SignatureAlg (M PK SK S : Type) extends OracleAlg spec where
    keygen : Unit → OracleComp spec (PK × SK)
    sign : PK → SK → M → OracleComp spec S
    verify : PK → M → S → OracleComp spec Bool
\end{leancode}
Any particular implementation is then required to specify how it intends the oracles to behave.
We also use this to define a simple type to represent security games:
\begin{leancode}
  structure SecExp (α : Type) extends OracleAlg spec where
    inpGen : OracleComp spec α
    main : α → OracleComp spec Bool
\end{leancode}
The advantage of an experiment can be defined by using the bundled simulation oracle to execute the experiment:
\begin{leancode}
  def advantage (exp : SecExp spec α β) : ℝ≥0∞ :=
    [= true | exp.exec (exp.inpGen >>= exp.main)]
\end{leancode}

As a simple example we have the following experiment for showing that a signing algorithm \lean{sigAlg} always verifies honest signatures:
\begin{leancode}
  def soundnessExp (m : M) : SecExp spec (PK × SK) Bool where
    inpGen := sigAlg.keygen ()
    main := λ (pk, sk) ↦ do
      let σ ← sigAlg.sign pk sk m
      sigAlg.verify pk m σ
\end{leancode}
Soundness is then expressed by saying that:
\begin{leancode}
  ∀ m : M, (soundnessExp sigAlg m).advantage = 1
\end{leancode}

For more complicated security games we need a representation of an adversary, and so we define a type \lean{SecAdv spec α β} for an adversary that takes a value of type \lean{α} and computes an output of type \lean{β} using oracles in \lean{spec}.
We also require that the adversary is polynomial time, and that we have an explicit bound on the number of queries they make.

For example in the unforgeability experiment for a signature algorithm, we have an adversary that has access to the regular oracles \lean{spec} of the protocol, and a signing oracle \lean{M →ₒ S} that that allows them to query for a valid signature on any message.
The input generation for this experiment is just the keygen function, and the main function gives the adversary the public keys and gets back a message \lean{m} and signature \lean{σ}.
The experiment succeeds if this signature is valid and the message was never queried:
\begin{leancode}
  def unforgeableExp (adv : SecAdv (spec ++ M →ₒ S) PK (M × S)) :
      SecExp spec (PK × SK) where
    inpGen := sigAlg.keygen ()
    main := λ (pk, sk) ↦ do
      let ((m, σ), log) ←
        simulate (sigAlg.signingOracle pk sk) (adv.run pk) (emptyLog spec)
      let b ← sigAlg.verify pk m σ
      return (b && !(log.wasQueried m))
\end{leancode}

\subsection{Proving the security of Schnorr Signatures}
\label{sec:schnorr}
Recall that Schnorr signatures are defined over a cyclic group \lean{Γ} of order \lean{p}, and using a random oracle \lean{hash : M × Γ × Γ -> Fin(p)}. 
For generator \lean{G : Γ}, we generate a key pair by choosing \lean{x <- $[0..p]},  \lean{let P := x * G;} and \lean{return (P,x)}; to sign a message \lean{m : M}, we choose \lean{k <- $[0..p]}, and compute \lean{a := k * G}, set \lean{e := query hash (m,a,P)}, and \lean{s := e*x+k mod p} and return \lean{(s,e)}.
Security follows from the argument that if an adversary \lean{fadv} can forge a signature \lean{(s,e)} on some message \lean{m} with noticeable advantage, then we can ``rewind'' \lean{fadv} to the point of its query \lean{hash (m,a,P)}, return a different value \lean{ee}, and obtain another valid signature \lean{(ss,ee)}, still with noticeable probability.
Given these two valid signatures, we can compute the private key \lean{x := (ss - s) / (ee - e) mod p}, which corresponds to the discrete logarithm of the public key \lean{P}.

In order to formalize this argument, prove a version of the ``forking lemma''~\cite{bellare2006multi} that states that if \lean{fadv} succceds with noticeable probability, then rewinding the computation also succeeds with noticeable probability. 
We define a function \lean{fork} that randomly chooses a query index \lean{q}, and then constructs two \lean{seed} values that correspond to the results of all oracle queries when running \lean{fadv}, diverging after index \lean{q}; \lean{fork} runs \lean{fadv} on its input \lean{P} with both \lean{seed} values and aborts (returning \lean{none}) unless it obtains the required signatures.

We then prove in Lean the forking lemma as the theorem
\begin{leancode}
  theorem le_fork_advantage (fadv : ForkAdv spec α β i) (x : α) :
    let frk := [isSome | (fork fadv).run x]
    let acc := [isSome | fadv.run x] -- forgery probability
    let q : ℝ≥0∞ := fadv.queryBound i + 1 -- num. signing queries
    let h : ℝ≥0∞ := Fintype.card (spec.range i) -- num. hash queries
    (acc / q) ^ 2 - 1 / h ≤ frk := _
\end{leancode}
The full proof can be found in our \vcvio repository\cite{vcvio-github}.  The proof draws on Mathlib~\cite{mathlib} machinery and consists of XXX lines of code.  (COMPARE TO PAPER VERSION)

(FORK FIGURE)

\subsection{Comparison}

We note that prior frameworks do not have the expressive power necessary to express a proof of the forking lemma, but in some cases may provide better automation and usability.  
As a point of comparison, Table~\ref{tab:thecomparisontable} includes the number of lines of code required for several example cryptographic security proofs in \vcvio as well as other state of the art frameworks.  (CITE CryptHOL, elaborate on results)


\section{Framework Extensions}

In order to enhance the usability of the \vcvio framework by other crypgtographers, and its applicability to the postquantum and privacy-enhancing cryptographic constructions investigated in the next sections, we will need to introduce several additional elements to the framework. 
First, we will introduce a library of program structures, lemmas, and proof tactics designed to support the argument style used in modern cryptographic security proofs, especially the ``game hopping'' proof strategy.  
Second, we will introduce methods to model ``proofs of knowledge,'' a common element used in modern cryptographic protocols, that are usually proven secure using rewinding lemmas.
Finally, we will introduce \vcvio structures to represent the hard lattice-based problems that form the basis for both homomorphic encryption schemes and many proposed post-quantum cryptosystems, along with structures to represent the homomorphic properties of these problems.

\subsection{Game Hopping} \label{sec:games}

(GAME HOPPING FIGURE: 4 HOPS)

% EasyCrypt seems to have the best current capability for this kind of argument
% not sure if "We will adapt the approaches taken there to our different underlying framework" sounds better though
% maybe even something like "EasyCrypt uses a custom language (rather than shallow-embedding) which supports much of this type of reasoning already. We expect that Lean4's very strong meta-programming abilities will make it possible to acheive similar results in our shallow-embedding as well"
Many cryptographic security proofs use the so-called ``game hopping'' or ``sequence of games'' argument strategy.
In this framework, the proof starts with a security experiment that defines the requirements for a protocol or primitive.
The proof then defines a sequence of modifications to this experiment, successive probabilistic ``games,'' along with arguments that the output distribution of successive pairs must be small, e.g. bounded by some concrete event probability or leading to a reduction to some hard problem, as shown in Figure~\ref{fig:game}.
The final game in the sequence has an easily characterized output distribution, for example, succeeding with probability $\frac{1}{2}$ or $\frac{1}{|X|}$ for some large set $X$.
Bellare and Rogaway~\cite{bellare2006security}, Shoup~\cite{shoup2004sequences}, and Halevi~\cite{halevi2005plausible} have all written tutorials in this style, along with a variety of useful lemmas about the types of game steps, and suggestions that this structure could be the basis of tools for formal verification of proofs.

% Also situations where the main function changes? i.e. make a small error check at the end of the "main" function
In \vcvio a game is essentially an instance of a \lean{SecExp} structure, where the \lean{SimOracle} structures used in simulating an adversary are successively modified.
% Is there a specific game thing you had in mind here or should I just add some example from one of those papers?
For example, Game (X FIXME) as shown in Figure~\ref{fig:game} could be implemented from Game (X-1 FIXME) using LEAN THING.
The project will implement the lemmas proposed in previous works~\cite{bellare2006security,shoup2004sequences,halevi2005plausible}, so they are available for use by other cryptographers.
Furthermore, the project will implement meta-tactics related to the range of intermediate game steps used in these works.

Throughout the project, the application of these tools to implement the security proofs in other phases of the project will serve as an iterative evaluation process.
If a game-hopping proof is required and can be completed using the mechanisms implemented in this phase, then the proof is a demonstration of the expressiveness of the existing library.
On the other hand, if in another phase of the project a game-hopping proof step cannot be easily expressed using the existing tools, then we will attempt to identify an opportunity to generalize the library further with additional lemmas or proof tactics.

\subsection{Proofs of Knowledge} \label{sec:pok}

A wide variety of modern cryptographic algorithms, for example ring signatures, anonymous credentials, contact tracing, and many cryptocurrencies, are based on noninteractive proofs (or arguments) of knowledge, or PoKs.  These typically serve to show that some value in a protocol is properly constructed or could only be produced by someone with access to a secret; flawed PoK protocols can lead to extreme security failures, e.g. the Plonk cryptocurrency was shown to be vulnerable to arbitrary forgeries due to an incorrect argument of knowledge implementation.

% If I add sigma protocols + F-S to the previous section then I guess bringing this with would make sense?
The common way to construct a PoK is by constructing a $\Sigma$-protocol for a hard relation $R(x,w)$, which consists of a pair of {\em prover} functions \lean{cmt, prv}, a {\em verifier} function \lean{ver}, a {\em challenge} space \lean{Chal}, a {\em simulator} function \lean{sim}, and an {\em extractor} function \lean{ext}.
The $\Sigma$-protocol is {\em complete} if for every \lean{c : Chal}, \lean{ver x (prv x w (cmt x w) c) = True};
it is {\em zero-knowledge} if \lean{sim x} is computationally indistinguishable from \lean{prv x w (cmt x w) c} for uniformly chosen \lean{c : Chal};
and it has ``special soundness'' if, when \lean{cmt <- (cmt x w)} we have $R(x,\lean{ext (prv x w cmt c1) (prv x w cmt c2)})$, that is, given the output of \lean{prv} for the same committed value \lean{cmt} and two different challenge values \lean{c1 c2 : Chal}, \lean{ext} can produce a witness for $x$.
The PoK is created from the $\Sigma$-protocol by the Fiat-Shamir heuristic of computing a challenge based on the output of a random oracle applied to \lean{(x, (cmt x w))}.

% "We will extend the current implementation of sigma protocols with Camenisch-Stadler techniques..."
Many $\Sigma$-protocols are constructed as conjunctions and disjunctions of $\Sigma$ protocols for simpler languages, using techniques introduced by Camenisch and Stadler~\cite{camenisch1997proof}.
Camenisch and Stadler also introduced the most prevalent notation for protocols constructed in this manner, e.g $PK(w_1,w_2)\{ R_1(x,w_1) \land R_2(x,w_2)\}$, indicating a proof of knowledge that $(x,w_1)$ satisfy some property and $(x,w_2)$ satisfy some other property.
% Is it worth discussing implementations in the previous section given it's mentioned here? At least basic implementation now with IO monad
The project will extend the \vcvio shallow embedding to incorporate a version of the Camenisch-Stadler "proof of knowledge" notation and composition techniques, allowing us and other developers to specify and instantiate composed proof of knowledge primitives from simpler primitives.
The resulting structures will both allow formally verifiable proofs of correctness and security based on the underlying primitives and (non-performant) formally-verified correct implementations.

% sigma protocols in vcvio are modeled as ...
We can model a basic $\Sigma$-protocol as a Lean structure specifying the computations and types mentioned above, where the relation \lean{rel} is simply a predicate \lean{rel : X -> W -> Bool}, the field \lean{Chal : Type}, and the various algorithms are instances of \lean{OracleComp} for the appropriate oracle specification and input and output types.
% extend the current model of sigma protocols to allow...
The project will use metaprogramming facilities in Lean to allow the specification of compound PoKs using a Camenisch-Stadler like notation.
We note that proving the security of cryptographic algorithms based on PoKs implicitly requires the use of rewinding techniques, so producing complete, formally verifiable security and correctness proofs of such algorithms is beyond the expressive ability of previous foundational systems.

As noted in section~\ref{sec:games}, the abstractions and artifact produced in this phase of the project can be subject to an iterative evaluation process.  The project will apply these tools in other phases of the project, for example in proving the security of a particular cryptographic algorithm.  If the resulting formal proof is successful and closely resembles the structure of the ``on paper'' proof, this serves as evidence of a successful implementation.  If the notational tools are not sufficient, then this will serve as an opportunity to revisit and expand the library of available PoKs and related meta-programs.

\subsection{Homomorphic primitives}
\label{sec:lattice}
In the past 15 years, many new cryptographic algorithms have been proposed based on hard lattice problems.  
These include many variants of the Learning with Errors (LWE) problem (Ring, Module, Rounding, Decision) and the Short Integer Solution (SIS) problem.  
Some interesting aspects of these problems is that compared to problems like discrete logarithm or integer factoring, they do not have unique solutions; they have a wider variety of parameters, including (potentially multiple) noise distributions, integer and polynomial moduli, and degrees and dimensions; they can lead to encryption or signing algorithms that produce errors; and the problem instances have useful algebraic structures.  
This in turn has led to interesting new primitives, such as the BGV~\cite{bgv} and CKKS~\cite{ckks} ``leveled homomorphic encryption'' schemes, which  form the basis of a variety of privacy-preserving protocols~\cite{,,} FIXME and post-quantum cryptosystems.

% I think \lean{HardRelation} is probably the model for these things. I could be wrong but I think hardness of e.g. LWE problems all would be instances of that typeclass
% I could add something like "Currently we have abstract hard relations, but only particular implementations of this for DL/DDH. The project will extend this with a more comprehensive set of problems"
To produce formally verifiable proofs of the correctness and security of schemes based on these problems, the project will introduce new Lean structures based on the \lean{OracleAlg} and \lean{SimOracle} types, which model the diversity of hard problems in this space.
The project will investigate and design \lean{SecExp} variants in \vcvio that express the security problems related to these primitives.  
The use of non-uniform distributions such as the discrete gaussian distribution, and the need for unusual algebraic structures will be notable challenges.
However, we expect that much of the needed analytical tools can be adapated from the Lean MathLib project.

% More details
\section{Postquantum cryptosystems}\label{sec:pq}

The NIST PQCRYPTO competition selected several algorithms for standardization as post-quantum cryptographic primitives.  While all of the selected algorithms have security proofs based on reductions to hard problems such as the short integer solutions problem, the module learning with errors problem and the security of cryptographic hash functions, the security proofs can be quite involved, and throughout the process of the competition several subtle errors in the proofs of the constructions have been identified and fixed.
It would be ideal if we had formally-verifiable proofs of the security of these constructions, but since the proofs involve the random oracle model and typically involve adversaries with access to multiple oracle types, such proofs are out of reach of existing frameworks.
Because \vcvio was designed to handle such obstacles, the project will aim to produce the first formally-verified security proofs for the SPHINCS+, Falcon, and Dilithium signature schemes.

%ML-KEM: MAYBE NOT?
%, and ML-KEM post-quantum cryptographic constructions

\subsection{SPHINCS+}\label{sec:sphincs}

SPHINCS+~\cite{bernstein2019sphincs+} is a hash-function based signature scheme that composes a "hypertree" of Merkle Tree, Winternitz One-time Signature (WOTS+)~\cite{hulsing2013w,merkle1989certified} and Forest of Random Signatures (FORS) signing schemes.
These schemes are composed using a pseudorandom function to generate the secret keys at different levels of the tree.
At a high level, a SPHINCS+ signature consists of a chain of WOTS+ signatures on Merkle Trees, whose leaves hold additional WOTS+ public keys, which are used to sign additional levels of Merkle Trees; at the lowest level, the Merkle Tree leaves are FORS public keys.
The FORS public key is then used to sign an actual message.
Each of these signatures works be revealing the pre-image of multiple cryptographic digests; signing a message involves hashing the message to a path in the tree, signing the tree roots along the path with the successive public keys, then signing a (separate) cryptographic hash of the message with the FORS leaf key.

\textbf{ADD FIGURE?}

Proving security of the FORS signature scheme requires modeling the hash function as a random oracle, and the concrete security of the scheme also requires reasoning about collisions in a truncated hash function.  The recursive structure of SPHINCS+ makes it a good fit to the inductive proof style of Lean, while the use of multiple oracles and concrete bounds, as well as the game-hopping nature of the security proof are well-suited to the extensions of the \vcvio framework to be developed in Thrust 1.

The project will proceed with a modular approach to proving the security, defining \lean{SecExp} structures that capture the different security and correctness properties required of the various signing schemes.
We will prove the security of SPHINCS+ using these assumptions, then prove that the WOTS+ and Merkle Tree schemes satisfies the required properties based on standard security assumptions about the cryptographic hash functions employed.
The \vcvio oracle-based formalism will allow us to prove the required security properties of the FORS scheme in the random oracle model.
An important aspect of the evaluation of the project will be the extent to which the \vcvio framework extensions support and help automate the security proof of SPHINCS+.


\subsection{Dilithium}

The Dilithium signature scheme~\cite{dilithum}, soon to be standardized as ML-DSA~\cite{ML-DSA}, is an instance of the ``Fiat-Shamir with aborts'' pattern.
Dilithium has undergone several revisions during the NIST PQC process; the FIPS ML-DSA standard is a slight revision of Dilithium 3.1, which required a change in an input length to fix a flaw in the security proof.
Due to the nature of the scheme, using the  Fiat-Shamir paradigm, a formally verifiable security proof has been out of reach of previous foundational cryptography frameworks.
% Paper recently that already did this: https://eprint.iacr.org/2023/246
% Key thing is that they can give a Fiat-Shamir for Identification protocols (although not sigma protocols w/o forking lemma)
% And showing the id-protocol security doesn't need forking (unlike the id protocol corresponding to Schnorr)
This project will construct the first formally verifiable foundational proof of the security of Dilithium.

Dilithium constructs a Schnorr-like $\Sigma$-protocol based on knowledge of a (M)LWE secret.
In this protocol, the secret key is a pair of short vectors (of polynomials) $s_1$, $s_2$, and the public key is a matrix (of polynomials) $A$, and the vector of polynomials $t = s_1A + s_2$.
The ``commit'' move of the protocol is to choose a vector $y$ and output $C = yA$; the challenge chooses a random polynomial $c$, and the response is $z = y + s_1c$.
The response is verified by checking that $z$ is short, and that $zA - ct$ is close to $yA$.
This $\Sigma$-protocol is converted to a signature scheme by the Fiat-Shamir transformation, adding an extra ``rejection sampling'' step that aborts if the response $z$ would leak information about the secret $s_1$.
% Side issue but rejection sampling is actually somewhat hard in the current approach I have. Syntactic abort would help
Additional optimizations to reduce the length of public keys and signatures are introduced, including several in the ML-DSA standardization to provide ``Beyond Unforgeability''~\cite{BUFF} security features.

The project will model the full details of the ML-DSA protocol.
We will attempt to follow as closely as possible the security proof from~\cite{dilithium3}, using game hopping with both standard reductions based on cryptographic hardness as well as probabilistic and entropy arguments when required.
We expect that the full proof will require the rewinding lemma from section~\ref{sec:schnorr}, many of the game hopping techniques from section~\ref{sec:games}, proof of knowledge definitions from section~\ref{sec:pok}, and lattice hardness assumptions from section~\ref{sec:lattice}.


\subsection{Falcon}\label{sec:falcon}

The Falcon~\cite{fouque2018falcon} signature scheme is the third signature scheme selected by NIST for standardization.
Falcon is based on the GPV~\cite{GPV} hash-and-sign lattice signature framework, relying on the hardness of the Short Integer Solution (SIS) problem in NTRU lattices.
The implementation of Falcon requires floating-point arithmetic and complex numbers, which are not well-supported in protocol-centered tools.
On the other hand, Lean has an extensive library of mathematical tools dealing with real and complex numbers,  MathLib, which will aid considerably in formally modeling and verifying the security and correctness of Falcon in \vcvio.

In Falcon, the public parameters include a dimension $n$, which is a power of $2$ and a prime modulus $q$ such that $n | (q-1)$.  Key generation involves choosing short polynomials $f,g,F,G \in \mathbb{Z}[x]/(x^n+1)$ such that $fG-gF = q \bmod (x^n+1)$, and computing the polynomial $h=gf^{-1} \bmod (x^n+1,q)$; the short polynomials form the signing key while the polynomial $h$ is the verification key.
To sign a message $m$, one computes a salted hash $c = H(r||m)$, represented as a polynomial in $\mathbb{Z}[x]/(x^n+1,q)$, then computes short polynomials $s_1,s_2$ such that $s_1 + s_2h = c \bmod (x^n+1,q)$; the signature is pair of values $r$ and $s_2$.
Verification recomputes $c$ from $m,r$ and checks that both $s_2$ and $c-s_2h \bmod (x^n+1,q)$ are short vectors.
Signatures are computed using the fact that $f,F,g,G$ can be used to construct a short basis for the lattice spanned by $\{1,h\}$, and applying a randomized rounding algorithm to prevent signatures from leaking the secret polynomials used to compute shortest vectors in this lattice.
This randomized procedure is optimized for fast signing using the Fast Fourier transform (FFT) and a recursive tree structure of decomposed representations of the basis.

The project will model the complete Falcon proposal, include any modifications made as the proposal transitions to a FIPS standard, using \vcvio.  We will attempt to follow as closely as possible the security proof from~\cite{GPV} as modified in~\cite{fouque2018falcon} for the NTRU structure, using game hopping with both standard reductions based on cryptographic hardness as well as probabilistic and geometric arguments when required.
We expect that the full proof will require many of the game hopping techniques from section~\ref{sec:games}, lattice hardness assumptions from section~\ref{sec:lattice}, and possibly additional techniques.
Note that the recursive structure of the key generation and signing algorithms~\cite{fouque2018falcon} lend themselves well to formalization in a shallow, functional embedding like \vcvio, in contrast to the more imperative-oriented embeddings of tools like EasyCrypt and FIXME.


%\subsection{ML-KEM?}
%Kyber, or ML-KEM, is a public-key encryption scheme based on the Module Learning With Errors problem.  ML-KEM employs a variant of the Fujisaki-Okamoto transform to convert a "one-way encryption scheme" into a chosen-ciphertext secure encryption scheme in the Random Oracle model.  HISTORY ABOUT FIXING THIS PROBLEM.

% Discussion of vcvio proof.

\section{End-to-end Encrypted Messaging}\label{sec:e2ee}
Many major messaging platforms -- WhatsApp, iMessage, Google messages, Telegram, Signal -- offer end-to-end encrypted messaging as an option for some communications, and standards or de-facto standards such as Noise~\cite{}, signal~\cite{}, RCS~\cite{} and MLS~\cite{} are being developed to potentially allow interoperability between these platforms.  Because of this widespread adoption and interest, there have been efforts to formally analyze or verify the security of some of these standards, particularly MLS, although these are primarily at the level of protocols and symbolic analysis rather than the cryptographic algorithms. CITES

However, these services do not necessarily conceal metadata about who communicates with whom, may offer or target different notions of deniability, and provide limited or no support for moderation.  
Many cryptographic protocols in the academic literature have been proposed to address these shortcomings, with some seeing adoption, such as Sealed Sender messages~\cite{} and Private Groups~\cite{} in Signal, and Message Franking in WhatsApp~\cite{}.  
However, there has been no effort to produce formally-verified proofs of the security or correctness of these protocols.  
As they move toward possible adoption at internet scale, verifying their security properties before widespread deployment is critical.

\subsection{Metadata concealment}\label{sec:omr}
An example of a flawed protocol currently in deployment is Signal's Sealed Sender messages~\cite{}. 
Sealed Sender messages attempt to defend against compromise of the Signal messaging servers by encrypting messages in an additional encrypted ``envelope'' that conceals the sender of the message from the server, so that in principle only the recipient can recover this information.  
However, Martiny {\em et al.}~\cite{} showed in 2021 that due to the application-level behavior of Signal, this information can be statistically recovered, allowing a compromised server to infer the probable originator of sealed sender messages.  
The PI and students~\cite{} later showed that these problems extend to group messages, revealing the participants of group conversations despite the use of Signal's Private Groups feature.
A root cause of these attacks is that the Signal server can observe which sealed-{\em sender} messages are retrieved by each {\em recipient}.

One class of protocols proposed to address this problem is {\em Oblivious Message Retrieval}, or OMR~\cite{}.  
OMR protocols allow a sender to submit messages to the server along with an encrypted "tag" that indicates the proper recipient.  
Recipients then submit an encrypted "query" that homomorphically aggregates the messages tagged for the recipient in a way that prevents the server from discovering which messages are aggregated.  
A series of recent works~\cite{} have developed efficient OMR protocols, but because the security definitions for these schemes involve leveled homomorphism, multi-oracle games, and in some cases proofs of knowledge, no formally verified proofs for their security exist.  
Since recent works have also identified some gaps in the proofs of previous schemes~\cite{snake}, having a framework for the formal verification of the correctness and security properties provided by these schemes will be an important step in promoting their adoption.

This project will investigate the security and correctness definitions for OMR protocols.  
We will seek to develop a formally verifiable security proof for at least one OMR construction in \vcvio, for example PerfOMR~\cite{}, which uses the BGV~\cite{BGV} cryptosystem for tag encryption. 
These efforts will make use of the game-hopping infrastructure developed in section~\ref{sec:games} and the lattice-based primitives and homomorphic security and correctness definitions developed in section~\ref{sec:lattice}.
We expect that even if other protocols in this family are ultimately adopted by messaging platforms, the results of our efforts will be helpful in constructing formally verifiable proofs for future protocols of a similar nature.

\subsection{Deniability}\label{sec:deniability}

Informally, deniability is the property that a protocol doesn't leak additional information binding a message to its sender.
This is often listed as a desirable property for an end-to-end encrypted messaging system~\cite{OTR,others...} CITES because adding confidentiality and authentication between users should not compromise the privacy of their conversations in other ways.

There have been many definitions of deniability in the literature on encrypted messaging~\cite{...}; most definitions require the existence of a forgery algorithm that would allow some party to produce a given transcript without communicating with the other participant(s) in the transcript.  
The variation inthe  inputs given to the forgery algorithm, and both its interaction with and the inputs given to a ``judge'' that attempts to distinguish this forgery from a real transcript give rise to the variety of definitions. 
Indeed, some definitions of deniability are incompatible with others; some of the variations include online or offline deniability, adapative or nonadaptive compromise, participation or message deniability, and instantaneous or epochal deniability.
For example, the MLS group key exchange protocol is not instantaneously deniable because every message in the protocol carries a signature from its originator; however, it can support epochal deniability by dividing time into epochs, using a certified ephemeral verification key during each epoch, and publishing the signing key at the end of the epoch.

While proofs of the confidentiality and authentication properties of key agreement protocols in both pairwise settings -- such as Signal's X3DH~\cite{} and PQXDH~\cite{} -- and group settings -- such as SenderKeys~\cite{} and MLS/TreeKem~\cite{} -- have been formally verified, there has been no comparable effort to create formally verifiable proofs of deniability for these protocols.
It is widely claimed that these protocols provide deniability, and this claim may contribute to the popularity of the tools that implement the protocols, which have user bases ranging from tens of millions to billions of users.
Because of the complexity and variety of definitions involved, and previous literature identifying gaps in the proofs of deniability for some of these protocols, formally verifying their deniability will close an important gap between their perceived and verified privacy properties.

The project will implement the modular deniability definitions of Fiedler and Janson~\cite{fiedler-janson-PETS24} in \vcvio, and attempt to formally verify the deniability proof of PQXDH. the post-quantum key agreement implemented in Signal.  
The project will also extend these definitions to epochal deniability and formally verify the epochal deniability proof of TreeKEM given by Hülsing and Weber~\cite{}.  
These definitions are well-suited for implementation in \vcvio because they are parameterized by the multiple types of oracles given to the forger and the distinguisher, an interface that is captured well by the \lean{OracleSpec} abstraction in \vcvio.  
We expect that the final proofs will also make use of many of the game-hopping strategies developed in Section~\ref{sec:games};  implementing the security proof for PQXDH will also require using the lattice hardness assumptions developed in Section~\ref{sec:lattice}.
The result of these efforts will give us greater confidence in the privacy properties of these widely-used protocols, and provide a framework for future efforts in formalizing deniability properties.

\subsection{Abuse reporting} \label{sec:abuse}
End-to-end encrypted messaging allows parties to keep their conversations private from the platforms that deliver these messages.  However, this same privacy can make it difficult for users to report or seek help avoiding abuse of the system to the platform: if a ciphertext is encrypted so that only the recipient can read it, then proving that it corresponds to a given abusive plaintext can be challenging.  Several related notions for addressing this problem -- message franking~\cite{FBM,grubbs-crypto17}, message tracing~\cite{Tyagi}, originator tracing~\cite{Peale} -- have been proposed in the literature, along with a variety of primitives for achieving these goals~\cite{SMF,AMF, Tyagi, Hecate, Cerberus, Peale, ACGKA, NDSS paper, PoPETS paper,MlsGov}.

SMF FIGURE

As an example, we consider two early schemes.  The first scheme, shown in Figure~\ref{}, is a Symmetric Message Franking (SMF) scheme, implemented in the encrypted Facebook Messenger system used by several hundred million people on a daily basis.
In this scheme, to send a message $m$ to recipient $r$, the sender $s$ chooses a random symmetric key $K_f$, then computes a tag $t = HMAC(K_f,m)$ and ciphertext $c = Enc(K_f||m)$ (encrypted using a symmetric key shared between the sender and the recipient), then sends $c,t$ to the server $F$.  
The server computes a second tag $\tau = HMAC(K_F,s||r||t)$ and sends $c,t,\tau$ to $r$.  
The recipient verifies that the tag $t$ is correct after decrypting $m$, and can report an abusive message by sending $s,r,m,K_f,t,\tau$ to the server.
Noting that this scheme requires the server's involvement and knowledge of sender and recipient, Tyagi {\em et al.}~\cite{AMF} proposed an Asymmetric Message Franking scheme (AMF) that replaces the tags $t$ and $\tau$ with a signature proof of knowledge on the plaintext that can only be computed with knowledge of the sender's secret key or the moderator's secret key.

An important goal of both the Facebook Messenger scheme and AMF, and many of the following schemes, is deniability: neither the recipient nor the judge should be able to convince any other party that the sender composed a particular message.  
Following work on deniable messaging, Tyagi {\em et al}~\cite{AMF} expressed this condition as requiring forger algorithms given a variety of goals, and proved that AMF satisfied this goal, but did not give a formally verifiable proof.
The SMF scheme has not had a full proof of deniability; while this property is intuitively provided by the use of symmetric primitives, we note that an earlier deployment by Facebook did not provide accountability~\cite{}.
As more E2EE systems move to deploy moderation tools, it is thus critical to have formally verified proofs that these tools provide the required privacy and security properties.

As in the case of deniable messaging, expressing the security conditions and proofs for these protocols has been out of the reach of previous cryptographic formal verification frameworks, due to the complex multi-oracle games and the use of more complicated primitives such as proofs of knowledge.  
In this project, we will use \vcvio to examine and give the first formally-verifiable security proofs for SMF and AMF. 
This work will build on the deniability experiments built in Section~\ref{sec:deniability}, making full use of the extensions developed in Phase 1 (sections~\ref{sec:games} and \ref{sec:pok}).
We expect the results to both provide greater assurance of the privacy provided to billions of users by Facebook Messenger, as well as providing a critical framework for formally verifying the security of future protocols developed and deployed for moderation and governance of encrypted group messaging systems. 

\section{Project Timeline, Prior Support and Broader Impact}
The requested duration of the project is 3 years.  In {\em Year 1}, the project will focus on extensions to the \vcvio framework along with the post-quantum signature schemes detailed in section~\ref{sec:pq}.  We hope to release the framework for public use on github, along with these security proofs.

In {\em Year 2}, the project will address the inclusion of homomorphic primitives to \vcvio and investigate applications to metadata-concealing protocols and deniability.

In {\em Year 3}, the project will address the inclusion of proofs of knowledge in \vcvio and abuse reporting.

\subsection{Results from prior support}

The PI was most recently supported by NSF grant 18xxxxx, {\em SatC:Core:Small:fingerprinting BLAH}.
\\

\noindent{\bf Intellectual Merit}... Stuff...

\\
\noindent{\bf Broader Impact}... Stuff...

\subsection{Broader Impact}
The results of these efforts will be broadly disseminated, including
conference, workshop and journal presentations.  The PI regularly
teaches graduate and undergraduate courses in cryptography and security as well as undergraduate courses for majors and non-majors, and will integrate the results of the research into lectures, exercises, and projects. In addition, the PI also serves as Director of Undergraduate Studies for Computer Science \& Engineering at the University of Minnesota and thus regularly participates in recruiting events organized by the College of Science and Engineering, exposing local high school students to research in cryptography and privacy.

The security models and framework develoepd and published by this project is expected to enable further exciting research in formally verifiable cryptographic proofs as well; the PI's prior research on the Tor project led to the release of the Shadow~\cite{jansen2012shadow} simulator, which has been used by over a CHECK - dozen different research groups and referenced by CHECK - 80 papers.

The project will result in undergraduate and graduate student training in research as well.  The PI has a strong record of involving undergraduates in research, having worked with over 30 undergraduates in the last 10 years, many of whom have gone on to or are currently in the process of applying to graduate programs.  The proposed research involves both cryptographic and programming language techniques and thus students involved in the project will be exposed to cross-cutting, interdisciplinary study.



\newpage
\pagenumbering{arabic}
\renewcommand{\thepage} {D--\arabic{page}}
%\centerline{\Large References}
\bibliography{vcvio}
\bibliographystyle{plain}

\end{document}
