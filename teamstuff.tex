The requested duration of the project is 3 years.  In {\em Year 1}, the project will focus on extensions to the \vcvio framework along with the post-quantum signature schemes detailed in section~\ref{sec:pq}.  We hope to release the framework for public use on github, along with these security proofs.

In {\em Year 2}, the project will address the inclusion of homomorphic primitives to \vcvio and investigate applications to metadata-concealing protocols and deniability.

In {\em Year 3}, the project will address the inclusion of proofs of knowledge in \vcvio and abuse reporting.

\subsection{Results from prior support}

The PI was most recently supported by NSF grant 18xxxxx, {\em SatC:Core:Small:fingerprinting BLAH}.
\\

\noindent{\bf Intellectual Merit}... Stuff...

\\
\noindent{\bf Broader Impact}... Stuff...

\subsection{Broader Impact}
The results of these efforts will be broadly disseminated, including
conference, workshop and journal presentations.  The PI regularly
teaches graduate and undergraduate courses in cryptography and security as well as undergraduate courses for majors and non-majors, and will integrate the results of the research into lectures, exercises, and projects. In addition, the PI also serves as Director of Undergraduate Studies for Computer Science \& Engineering at the University of Minnesota and thus regularly participates in recruiting events organized by the College of Science and Engineering, exposing local high school students to research in cryptography and privacy.

The security models and framework develoepd and published by this project is expected to enable further exciting research in formally verifiable cryptographic proofs as well; the PI's prior research on the Tor project led to the release of the Shadow~\cite{jansen2012shadow} simulator, which has been used by over a CHECK - dozen different research groups and referenced by CHECK - 80 papers.

The project will result in undergraduate and graduate student training in research as well.  The PI has a strong record of involving undergraduates in research, having worked with over 30 undergraduates in the last 10 years, many of whom have gone on to or are currently in the process of applying to graduate programs.  The proposed research involves both cryptographic and programming language techniques and thus students involved in the project will be exposed to cross-cutting, interdisciplinary study.


