\section{Background and Preliminary work}

\subsection{Previous frameworks and challenges}
A wide variety of frameworks have been developed to reason about cryptographic proofs in the computational setting, which all have different benefits and drawbacks depending on the specific use cases.
Some like EasyCrypt \cite{canetti2019easyuc}, CryptoVerif \cite{blanchet2007cryptoverif} or SSProve \cite{abate2021ssprove} take a protocol centered approach that focuses mainly on high-level functionality and behavior, often using a deeply-embedded language and proof system to represent and reason about computations.
Others like FCF \cite{FCF,petcher2015foundational}, CertiCrypt \cite{zanella2010formal}, or CryptHOL \cite{basin2020crypthol} take a more foundational approach, building all constructions from a small computing base with limited axiomatization.
Usually computations in this approach are represented by a shallow embedding into a more general proof assistant like Coq or Isabelle.
The foundational approach has significant benefits in terms of flexibility and extensibility, and allows for concrete security bounds,
however it usually provides less automation when writing proofs.

One important limitation with existing foundational systems is how they reason about the oracles available to a computation.
In FCF, for example, oracle access is specified by a single oracle with fixed input and output types. This makes it difficult to reason about situations where multiple (or varying) oracles are available, e.g. both a signing oracle and a random oracle.  Although dependent sum types for the domain and range of different oracles could be used, the specification of programs and reductions that use these oracles quickly becomes cumbersome, and cannot easily handle composition of programs that use different sets of oracles.
In CryptHOL, oracles are implemented as ``generative probabilistic values'' and require restructuring adversaries and experiments into a series of smaller computations that run between oracle calls.  Because adversaries cannot access the state of oracles, CryptHOL cannot easily express reductions in which oracles are reactively ``programmed'' or rewound to intermediate states by one adversary -- suach as an adversary that solves a hard computational problem -- in response to the queries or results output by another adversary, such as an adversary that outputs a signature forgery. 

\subsection{The VCVio framework}
\label{sec:vcvio}
Similar to FCF and CryptHOL, \vcvio uses a shallow embedding of cryptographic algorithms, reductions, and security definitions as functional programs in the Lean proof assistant.  Lean  is a dependently typed programming language and interactive proof assistant that enables computer verification of both mathematical proofs and properties of programs, similar to Coq or Isabelle.  

At a basic level Lean is a functional programming language, and its syntax is similar to languages like Haskell or OCaml.
The core logic of Lean is based on a type theory called the calculus of constructions (CIC), where every expression is a term and every term has a type, denoted \lean{(a : A)}.
The type of a function from type \lean{A} to type \lean{B} is written \lean{A → B}, and function terms are expressed using lambda notation:

\begin{leancode}
  def foo : ℕ → ℕ → ℕ := λ x y ↦ x * y + 1
\end{leancode}

Functions can be dependently typed, so the type of an argument (or the output type) can depend on previous arguments to the function.
For example the function \lean{List.cons} is polymorphic in the type of the list's elements, which we express by naming the type of the first argument:

\begin{leancode}
List.cons : {A : Type} → List A → A → List A
\end{leancode}

The curly braces indicate that when this function is called Lean should try to automatically determine the value of \lean{A} based on type inference. 
This allows us to write \lean{List.cons xs x} whether the elements of \lean{xs} are strings, natural numbers, or something else.

The standard algebraic data types are built into the main language.
Product types are written \lean{A × B} with canonical elements \lean{(a, b)} for \lean{a : A} and \lean{b : B}.
Sum types are written \lean{A ⊕ B} with canonical elements \lean{inl a} and \lean{inr b}.
The singleton type \lean{Unit} has a single canonical element \lean{()}, and the empty type \lean{⊥} has no elements.
We also have a type \lean{Bool}, with canonical elements \lean{true} and \lean{false}.

In order to be able to write and verify proofs, Lean also has a type \lean{Prop} to represent mathematical propositions, and we view the propositions in \lean{Prop} as themselves being types.
Under this framework, an element \lean{P : Prop} is a mathematical statement, and the "elements" of \lean{P} are the proofs of that statement.  While it's possible to write proofs as explicit terms in the language, lean also offers an interactive proof environment that allows proofs to be written via \textit{tactic programming}.
A tactic proof consists of a number of steps that modify the current "goal" to be proven, either solving it explicitly or generating new goal(s) that would suffice to complete the proof.
The initial goal is the statement being proved, and a proof is complete if no goals remain.
Many tactics are built in to the core language, but they can also be defined by users.
One particularly useful tactic is \lean{simp}, which takes in a list of equivalences, and attempts to replace any instances of the left hand side of the equality in the goal with the right hand side. 
As an example we can use it to show that if $x + y = z$ for some non-negative $y$, then either $x < z$ or $y$ is $0$:

\begin{leancode}
  example (x y z : ℝ) (hy : 0 ≤ y)
      (hz : x + y = z) : x < z ∨ y = 0 :=
    by simp [← hz, le_iff_lt_or_eq, hy]
\end{leancode}

\subsubsection{A Monad for Computations with Oracle Access} \label{OracleComp}
In order to reason about cryptographic protocols, we define a shallow embedding of computations with oracle access into the Lean type system, using a free monad to augment regular Lean functions with queries to oracles.

\paragraph{Specifying Oracles}
Before defining our model of computation we define a way to specify the set of oracles available to a computation, via a structure \lean{OracleSpec} consisting of an indexing set $\iota$ of available oracles and maps from indices to the domain and range types of the corresponding oracle:
\begin{leancode}
  structure OracleSpec where (ι : Type) (domain range : ι → Type)
\end{leancode}

Each element \lean{i : ι} of the indexing set corresponds to a unique oracle, and \lean{domain i} and \lean{range i} are the input and output types of the oracle corresponding to \lean{i}.
Note that we don't specify any behavior for these oracles; this should rather be seen as analogous to a type signature for them.
For instance we define \lean{singletonSpec T U} to represent access to
a single oracle with input type \lean{T} and output type \lean{U},
by using the singleton type \lean{Unit} for the indexing set, and have the \lean{domain} and \lean{range} functions return \lean{T} and \lean{U} respectively:
\begin{leancode}
  def singletonSpec (T U : Type) : OracleSpec :=
    (ι := Unit) (domain := λ () ↦ T) (range := λ () ↦ U)
\end{leancode}
We also define \lean{emptySpec : OracleSpec} to represent having no oracles at all.
We introduce the notation \lean{T →ₒ U} for the singleton spec and \lean{[]ₒ} for the empty spec.

In order to represent probabilistic computation we define two additional oracle sets.
Firstly \lean{coinSpec} that gives access to a single coin flipping oracle returning a \lean{Bool},
and secondly \lean{unifSpec} that gives access to an infinite family of oracles, indexed over \lean{ℕ}, where the $n$th oracle chooses a random value between $0$ and $n$ inclusively (represented by the type \lean{Fin (n + 1)} in Lean).
Since the input to these oracles is irrelevant, in both cases we use \lean{Unit} for the domain types:
\begin{leancode}
  def coinSpec : OracleSpec := Unit →ₒ Bool
  def unifSpec : OracleSpec :=
    (ι := ℕ) (domain := λ _ ↦ Unit) (range := λ n ↦ Fin (n + 1))
\end{leancode}

\paragraph{Representing Computations.}
We now define a way to represent computations via an inductive type \lean{OracleComp spec α}, where \lean{spec : OracleSpec} specifies what oracles the computation can make use of, and \lean{α} gives the type of the final output.
This type is a shallow embedding, so functions and expressions in this representation are just functions and expressions in Lean, augmented with the ability to call oracles.

The embedding is done as a monad, which is a common way to represent computations with side effects in languages where all computations are pure.
In our case the "side effects" being captured by the monad are queries to the oracles.  We define this as an inductive type with only two constructors.
The first is \lean{pure' α x} which represents returning an pure Lean value \lean{x : α}.
The other, \lean{queryBind' i t α oa}, represents querying oracle \lean{i} on input \lean{t} to get a result \lean{u} and then running the computation \lean{oa u}.  Explicitly:
\begin{leancode}
  inductive OracleComp (spec : OracleSpec) : (α : Type) → Type
    | pure' (α : Type) (x : α) : OracleComp spec α
    | queryBind' (i : spec.ι) (t : spec.domain i)
        (α : Type) (oa : spec.range i → OracleComp spec α) :
        OracleComp spec α
\end{leancode}
Note that the function \lean{oa} in the \lean{queryBind'} constructor is an arbitrary Lean function.
It can therefore perform arbitrary computations in the process of returning the continuation.
Most of the "interesting" behavior of a computation is captured in this function, with the monad structure simply adding the ability to represent query calls.

We define \lean{query i t} to be the computation that returns the result of querying oracle \lean{i} with \lean{t}, directly returning the result as a pure value after:
\begin{leancode}
  def query (i : spec.ι) (t : spec.domain i) : 
    OracleComp spec (spec.range i) :=
    queryBind' i t (spec.range i) (λ u ↦ pure' (spec.range i) x)
\end{leancode}
We define computations representing flipping a coin and selecting from a range as special cases of \lean{query}:
\begin{leancode}
  def coin : OracleComp coinSpec Bool := query () ()
  def $[0..n] : OracleComp unifSpec (Fin (n+1)) :=  query () n
\end{leancode}
We overload notation and write \lean{$ xs} for random selection from any type of collection.

The general monadic bind operation on this type, \lean{bind' α β oa ob} will represent running the computation \lean{oa} to get a result \lean{x : α}, then running \lean{ob} on input \lean{x} to get a result of type \lean{β}.
We define this by induction on the first computation.
If the first computation \lean{oa} is a pure value, we insert it directly into the remaining computation.
If the first computation is a query bound to a continuation, we move the second computation inside the continuation by a recursive call to \lean{bind'}.
\begin{leancode}
  def bind' (α β : Type) : OracleComp spec α →
      (α → OracleComp spec β) → OracleComp spec β
    | pure' α a, ob => ob a
    | queryBind' i t α oa, ob =>
        query_bind' i t β (λ u ↦ bind' α β (oa u) ob)
\end{leancode}
\lean{OracleComp spec} with the operations \lean{pure'} and \lean{bind'} then forms a monad, and the three monad laws can all be verified easily by induction.
We will generally use Lean's monad type class and its associated notation to write computations.
In particular we can write \lean|return a| for the pure operation, \lean|oa >>= ob| for the bind operation, and use \textit{do}-notation for sequencing larger computations, as in \lean{example : OracleComp coinSpec ℝ := do let b ← coin;} \lean{let b' ← coin;} \lean{ return (if b && b' then 3.14 else 1.61)}.

\paragraph{Sub-Specs and Type Coercions}
In order to combine sets of oracles we define an append operation on \lean{OracleSpec}, to represent having access to oracles in either of the two original specs. We make use of sum types for the indexing set, with the \lean{inl} and \lean{inr} functions used to index the first and second sets of oracles respectively.
The types of the domain and range at each index are defined by pattern matching on the provided index.
%:
%\begin{leancode}
%  def append (spec₁ spec₂ : OracleSpec) : OracleSpec where
%    ι := spec₁.ι ⊕ spec₂.ι
%    domain := λ i ↦ match i with
%      | inl i => spec₁.domain i
%      | inr i => spec₂.domain i
%    range := λ i ↦ match i with
%      | inl i => spec₁.range i
%      | inr i => spec₂.range i
%\end{leancode}
We introduce the notation \lean|spec ++ spec'| for this operation.

One major issue in this representation is that it gives no way to sequence or combine computations where one has only a subset of the oracles of another.
For example we can't bind \lean{coin} to an adversary that has access to both a coin oracle and random oracle, since our monad definition requires that the \lean{OracleSpec} remain fixed throughout a computation.
To solve this we create a system for automatically coercing the type of a computation to one with a larger set of oracles.
We do this using Lean's \lean{Coe A B} type-class, which specifies to Lean a way to automatically convert a value of type \lean{A} to one of type \lean{B} (e.g. \lean{Coe ℚ ℝ} allows a rational number to be viewed as an element of the reals).
Lean will automatically perform a type-class search for this whenever it fails to type check an expression, and apply a coercion if it finds one.  
The semantics we define in the next to sections are highly compatible with these coercions, and and our system is generally able to reduce a proof about a coerced computation down to an analogous proof about the original computation automatically.

\paragraph{Probability Semantics} \label{DistSemantics}
We next give a denotational semantics for \lean|OracleComp|, where the denotation is a probability distribution modeling the probability of getting specific outputs from the computation.  Specifically, we define a function \lean{evalDist} that maps computations with return type \lean{α} to probability mass functions on \lean{α}, represented by the type \lean{PMF α} (originally developed for use in the CryptoLib project \cite{CryptoLib}).
\lean{PMF} itself is a monad and the mapping is a morphism of monads (i.e. it respects \lean{pure} and \lean{bind}).
It's then possible to define a measure on \lean{α} induced by this \lean{PMF}, and to define the probability of some predicate \lean{p} holding after running a computation as the measure of the set \lean{{x | p x}}.

We will write \lean{[= x | oa]} for the probability associated to \lean{x} by \lean{evalDist oa}, which gives a simple characterization of the semantics in practice:
\begin{leancode}
[= x | return a] = if x = a then 1 else 0
[= y | oa >>= ob] = ∑ x, [= x | oa] * [= y | ob x]
[= u | query i t] = 1 / (spec.range i).card
\end{leancode}
We also define \lean{[p | oa]} to be the probability associated to the event \lean{p : α → Prop}.
If the event is instead a set \lean{e : Set α} then we can write \lean{[(· ∈ e) | oa]} for the probability that the output is a member of \lean{e}. 
We define \lean{support oa} to be the set of possible outputs of \lean{oa}, i.e. the set of \lean{x} such that \lean{[= x | oa] ≠ 0}.
We make heavy use of this when showing that an event has probability either \lean|1| or \lean|0|, allowing us to think about ``possible outputs'' rather than explicit probabilities.

\paragraph{Simulation Semantics} 
We now give an alternative semantics for \lean{OracleComp}, which provide a method for simulating the behavior of oracles.
The main construction is a function \lean{simulate} that recursively substitutes the queries to an oracle with a specified implementation (that may have access to a different set of oracles).
We allow the simulation to maintain some internal state, and augment the return value of the computation with the final state.
For example, simulating a random oracle will consist of maintaining an internal cache of input/output pairs as the state, and the replacement will substitute queries to include a check to the cache before responding.

In order to represent a procedure for simulating a computation, we define a type \lean{SimOracle} \lean{spec} \lean{spec'} \lean{σ} for an implementation of the oracles in \lean{spec} using a new set of oracles \lean{spec'}, where \lean{σ} is the type of an internal state shared throughout the simulation.
This is given by a function that takes an oracle input and returns a new computation that should be used to replace the query:
\begin{leancode}
  def SimOracle (sp : OracleSpec) (sp' : OracleSpec) (σ : Type) :=
    (i : sp.ι) → sp.domain i → σ → OracleComp sp' (sp.range i × σ)
\end{leancode}
We introduce the notation \lean{spec →[σ] spec'} for this type.

We then define a function \lean{simulate} that applies a simulation oracle to a computation, recursively replacing queries with the new computations, passing the updated state along before returning the final value of the computation and the final state value.
\begin{leancode}
  def simulate (so : spec →[σ] spec') :
      (oa : OracleComp spec α) → (s : σ) → OracleComp spec' (α × σ)
    | pure' α x, s => return (x, s)
    | query_bind' i t α oa, s => do let (u, s') ← so i t s;
        simulate so (oa u) s'
\end{leancode}

In order to represent more complex oracle implementations in a modular way, we define a number of ways to combine multiple simulation oracles.
To construct a combined simulation oracle for \lean{spec₁ ++ spec₂}
given individual simulation oracles for each, the simulation oracle pattern matches the oracle index it receives and forwards to the corresponding simulation oracle, maintaining independent states for each of them.
We define the \lean{compose} simulation to construct a simulation oracle that applies two simulation oracles in sequence, by simulating the simulation function of one with the other.  We will write \lean{++} and \lean{∘} for these two operations.

Besides just implementing oracle behavior, another important use case
is tracking some information about a computation.
For example we can easily implement simulation oracles \lean{loggingOracle} for logging the queries made by an oracle, \lean{cachingOracle} for logging fresh values, but returning old values if the input has been queried already, and \lean{randOracle} as the composition of a \lean{cachingOracle} and a \lean{unifOracle}.

\subsubsection{Cryptographic Protocols and Security Games} \label{sec:protocols}
In this section we give a simple abstraction layer that can be used to specify cryptographic primitives, protocols, and security games.
We focus here on using this to establish concrete security bounds, however it can be directly extended to the asymptotic setting.
We start by defining a type \lean{OracleAlg spec} that just contains a specification of how to simulate oracles in \lean{spec} using \lean{unifSpec}.
Here \lean{spec} should be thought of as some global oracles for a protocol (e.g. a random oracle), and the structure as containing the ``intended behavior'' of them.
We also assume an intended initial state (e.g. an empty cache):
\begin{leancode}
  structure OracleAlg (spec : OracleSpec) where
   (baseState : Type) (init_state : baseState)
   baseSimOracle : spec →[baseState] unifSpec
\end{leancode}
Given \lean{alg : OracleAlg spec}, we will write \lean{alg.exec oa} as shorthand for simulating \lean{oa} with the bundled oracle in \lean{alg}. 
This allows us to define the type of a cryptosystem in a way that is agnostic to the oracles that are available.

As an example we define the type of a signature protocol as extending this structure with keygen, sign, and verify functions:
\begin{leancode}
  structure SignatureAlg (M PK SK S : Type) extends OracleAlg spec where
    keygen : Unit → OracleComp spec (PK × SK)
    sign : PK → SK → M → OracleComp spec S
    verify : PK → M → S → OracleComp spec Bool
\end{leancode}
Any particular implementation is then required to specify how it intends the oracles to behave.
We also use this to define a simple type to represent security games:
\begin{leancode}
  structure SecExp (α : Type) extends OracleAlg spec where
    inpGen : OracleComp spec α
    main : α → OracleComp spec Bool
\end{leancode}
The advantage of an experiment can be defined by using the bundled simulation oracle to execute the experiment:
\begin{leancode}
  def advantage (exp : SecExp spec α β) : ℝ≥0∞ :=
    [= true | exp.exec (exp.inpGen >>= exp.main)]
\end{leancode}

As a simple example we have the following experiment for showing that a signing algorithm \lean{sigAlg} always verifies honest signatures:
\begin{leancode}
  def soundnessExp (m : M) : SecExp spec (PK × SK) Bool where
    inpGen := sigAlg.keygen ()
    main := λ (pk, sk) ↦ do
      let σ ← sigAlg.sign pk sk m
      sigAlg.verify pk m σ
\end{leancode}
Soundness is then expressed by saying that:
\begin{leancode}
  ∀ m : M, (soundnessExp sigAlg m).advantage = 1
\end{leancode}

For more complicated security games we need a representation of an adversary, and so we define a type \lean{SecAdv spec α β} for an adversary that takes a value of type \lean{α} and computes an output of type \lean{β} using oracles in \lean{spec}.
We also require that the adversary is polynomial time, and that we have an explicit bound on the number of queries they make.

For example in the unforgeability experiment for a signature algorithm, we have an adversary that has access to the regular oracles \lean{spec} of the protocol, and a signing oracle \lean{M →ₒ S} that that allows them to query for a valid signature on any message.
The input generation for this experiment is just the keygen function, and the main function gives the adversary the public keys and gets back a message \lean{m} and signature \lean{σ}.
The experiment succeeds if this signature is valid and the message was never queried:
\begin{leancode}
  def unforgeableExp (adv : SecAdv (spec ++ M →ₒ S) PK (M × S)) :
      SecExp spec (PK × SK) where
    inpGen := sigAlg.keygen ()
    main := λ (pk, sk) ↦ do
      let ((m, σ), log) ←
        simulate (sigAlg.signingOracle pk sk) (adv.run pk) (emptyLog spec)
      let b ← sigAlg.verify pk m σ
      return (b && !(log.wasQueried m))
\end{leancode}

\subsection{Proving the security of Schnorr Signatures}
\label{sec:schnorr}
Recall that Schnorr signatures are defined over a cyclic group \lean{Γ} of order \lean{p}, and using a random oracle \lean{hash : M × Γ × Γ -> Fin(p)}. 
For generator \lean{G : Γ}, we generate a key pair by choosing \lean{x <- $[0..p]},  \lean{let P := x * G;} and \lean{return (P,x)}; to sign a message \lean{m : M}, we choose \lean{k <- $[0..p]}, and compute \lean{a := k * G}, set \lean{e := query hash (m,a,P)}, and \lean{s := e*x+k mod p} and return \lean{(s,e)}.
Security follows from the argument that if an adversary \lean{fadv} can forge a signature \lean{(s,e)} on some message \lean{m} with noticeable advantage, then we can ``rewind'' \lean{fadv} to the point of its query \lean{hash (m,a,P)}, return a different value \lean{ee}, and obtain another valid signature \lean{(ss,ee)}, still with noticeable probability.
Given these two valid signatures, we can compute the private key \lean{x := (ss - s) / (ee - e) mod p}, which corresponds to the discrete logarithm of the public key \lean{P}.

In order to formalize this argument, prove a version of the ``forking lemma''~\cite{bellare2006multi} that states that if \lean{fadv} succceds with noticeable probability, then rewinding the computation also succeeds with noticeable probability. 
We define a function \lean{fork} that randomly chooses a query index \lean{q}, and then constructs two \lean{seed} values that correspond to the results of all oracle queries when running \lean{fadv}, diverging after index \lean{q}; \lean{fork} runs \lean{fadv} on its input \lean{P} with both \lean{seed} values and aborts (returning \lean{none}) unless it obtains the required signatures.

We then prove in Lean the forking lemma as the theorem
\begin{leancode}
  theorem le_fork_advantage (fadv : ForkAdv spec α β i) (x : α) :
    let frk := [isSome | (fork fadv).run x]
    let acc := [isSome | fadv.run x] -- forgery probability
    let q : ℝ≥0∞ := fadv.queryBound i + 1 -- num. signing queries
    let h : ℝ≥0∞ := Fintype.card (spec.range i) -- num. hash queries
    (acc / q) ^ 2 - 1 / h ≤ frk := _
\end{leancode}
The full proof can be found in our \vcvio repository\cite{vcvio-github}.  The proof draws on Mathlib~\cite{mathlib} machinery and consists of XXX lines of code.  (COMPARE TO PAPER VERSION)

(FORK FIGURE)

\subsection{Comparison}

We note that prior frameworks do not have the expressive power necessary to express a proof of the forking lemma, but in some cases may provide better automation and usability.  
As a point of comparison, Table~\ref{tab:thecomparisontable} includes the number of lines of code required for several example cryptographic security proofs in \vcvio as well as other state of the art frameworks.  (CITE CryptHOL, elaborate on results)

